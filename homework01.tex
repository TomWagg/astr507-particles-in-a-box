\documentclass[12pt]{article}
%\usepackage{psfig}
\begin{document}

\noindent ASTR 507 Winter 2022: Homework 1, Due Jan.\ 20, 2022


\begin{enumerate}
\item Simulating an ideal gas:

Numerically compute the distribution of kinetic energy of a {\bf two}-dimensional monatomic ideal gas.  Start off with equal kinetic energy, $E$, for $N$ particles in a square box, but random positions and moving in random directions.  Follow the particles until they collide (be careful how you handle the collisions).  Conserve momentum and energy in the collisions (see, e.g., Reif, 14.1-14.2, Feynman 39) and assume that the colliding particles hard spheres (in which case they experience an elastic force directed along the line connecting the center of the spheres when they touch).   You may instead make the approximation that when they collide the particles receive a kick in a random direction in the center-of-mass frame (make sure you conserve energy and momentum!).  Allow the cross section and number of particles to vary in your program.  You can make the sides of the box reflecting or periodic.

(a) [6 pt] Write a detailed summary of what choices you make in your code and how it works. Specifically, address the dimensions you use for velocity, mass, energy, length, and time.  Discuss how you determine if a collision has occurred, what equations you solve during a collision, and how you handle the box boundaries. Email me your code, preferably written in Julia in a commented Jupyter notebook.

(b) [4 pt] Make four plots for a case with 100 particles, each with a radius equal to 2\% of the size of the box. The first two plots should show the position of the particles (and sizes if you would like) at the initial time, and another snapshot after a long enough time that the particle velocity distribution has reached a steady-state in which the initial condition is forgotten. The second two plots should show the velocity distribution of the particles at the initial and final times.

\item Maxwellian:

(a) [2 pt] Derive the energy and velocity distribution for a Maxwellian as a function 
of speed and energy {\bf in two dimensions}, $dN/dv$ and $dN/dE$, where
$v = \vert \mathbf{v} \vert|$ is the speed, $\mathbf{v}$ is the velocity,
and $E$ is the energy.  You can
start with the result that $f(q,p) \propto e^{-(E/k_BT)}$.

(b) [3 pt] Show that the energy (or speed) distribution 
of your simulation particles approaches a 2-D Maxwellian by
plotting the speed distribution, either a histogram or cumulative distribution, 
averaged over several timesteps (or over several simulations) to make a smooth
curve.  Overplot the 2-D Maxwellian distribution you computed in part (a),
normalized to agree.

(c) [3 pt] Now make half of the masses 10 times larger.
Does the velocity distribution follow a single Maxwellian? Does the
energy distribution?  Explain your results.

\item  Relaxation timescale:

Initially the distribution of particle energies changes rapidly, but
eventually it approaches a steady-state.  

(a) [2 pt] Invent a numerical criterion for when your system has ``relaxed," 
i.e.\ how can you tell from the distribution of particle velocities (or
energies) that the system is in a statistically steady-state. 

(b) [3 pt] Derive a formula for the relaxation time for each particle 
as a function of the initial velocity of each particle, $v_0$, the cross section,
$\sigma$ (which in 2-D case has units of length, not area), and the
number density of particles, $n$ (which has units of 1/area in 2-D).

(c) [3 pt] Vary the number density, particle size (i.e.\ cross section), and
number density of particles to show that the criterion you developed in part
(a) scales in the same way as the relaxation time from part (b).
Demonstrate that these relations agree with a plot or two.

\item  Extra credit: Pressure

[4 pt] Demonstrate that $P=nkT$ holds for your simulations - try a few
different values of $n$ and $kT$ (i.e.\ number density and mean energy of particles).
Compute the mean pressure by calculating the force per length (since 2-D)
on each (or just one) side of your box.  You can do this by using the
definition of pressure:  force divided by length (in 2-D), and compute
the forces of each particle as it reflects off of each side (if you have
periodic boundary conditions, then keep track of the momentum perpendicular
to the edges as the particles jump from one side of the box to the other).   
The average force imparted is the average change in momentum of the 
particle as they reflect (or twice the momentum perpendicular to the side) 
divided by the average time between collistions.
\end{enumerate}
\end{document}
